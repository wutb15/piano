





%%%%%%%%%%%%%%%%%%%%%%%%%%%%%%%%%%%%%%%%%%%%%%%%%%%%%%%%%%%%%%%%%%%%%%%%%%%
%%
%%  LaTeX + CJK 模板,只针对 A4 纸的中文Paper。
%%
%%  Ver 1.02 By DeathKing @ <dk.hit.edu.cn>
%%  Ver 1.01 By rabbitbug @ www.ctex.org
%%  Ver 1.0 by oLo @ bbs.ustc.edu.cn
%%
%%  You can mofify it and distribute it freely:)
%%
%%%%%%%%%%%%%%%%%%%%%%%%%%%%%%%%%%%%%%%%%%%%%%%%%%%%%%%%%%%%%%%%%%%%%%%%%%%%

%%%%%%%%%%%%%%%%%%%%%%%%%%%%%%%%%%%%%%%%%%%%%%%%%%%%%%%%%%%%%%%%
%  文章模板:A4 纸,小五字,单列(可根据要求改双列 twocolumn)
%%%%%%%%%%%%%%%%%%%%%%%%%%%%%%%%%%%%%%%%%%%%%%%%%%%%%%%%%%%%%%%%
\documentclass[a4paper,11pt,onecolumn,twoside]{article}
\usepackage{CJK}         
\usepackage{fancyhdr}
\usepackage{amsmath,amsfonts,amssymb,graphicx}    
\usepackage{subfigure}   
\usepackage{indentfirst} 
\usepackage{bm}          
\usepackage{multicol}    
\usepackage{indentfirst} 
\usepackage{picins}      
\usepackage{abstract}    
\addtolength{\topmargin}{-54pt}
\setlength{\oddsidemargin}{-0.9cm}  
\setlength{\evensidemargin}{\oddsidemargin}
\setlength{\textwidth}{17.00cm}
\setlength{\textheight}{24.00cm}    


\renewcommand{\baselinestretch}{1.1}
\parindent 22pt 

\begin{CJK}{UTF8}{gbsn}
\title{\huge{Matlab 钢琴作曲}}
\author{吴田波 荆潇\\[2pt]
\normalsize
(清华大学自动化系,自53班~学号~2015011428 自53班~学号~2015011431) \\[2pt]}
\date{}  
\end{CJK}


\fancypagestyle{plain}{
\fancyhf{}
\lfoot{}
\cfoot{}
\rfoot{}}


\pagestyle{fancy}
\fancyhf{}
\fancyhead[LE,RO]{\thepage}
\lfoot{}
\cfoot{}
\rfoot{}

\newenvironment{figurehere}
  {\def\@captype{figure}}
  {}
\makeatother

\usepackage{titlesec}
\newcommand{\sectionname}{节}
\renewcommand{\figurename}{图}
\renewcommand{\tablename}{表}
\renewcommand{\bibname}{参考文献}
\renewcommand{\contentsname}{目~录}
\renewcommand{\listfigurename}{图~目~录}
\renewcommand{\listtablename}{表~目~录}
\renewcommand{\indexname}{索~引}
\renewcommand{\abstractname}{\Large{摘~要}}
\newcommand{\keywords}[1]{\\ \\ \textbf{关~键~词}:#1}
\titleformat{\section}[block]{\large\bf}{\thesection}{10pt}{}




\begin{document}
\begin{CJK*}{UTF8}{gbsn}
\newcommand{\supercite}[1]{\textsuperscript{\cite{#1}}}
\maketitle

\setlength{\oddsidemargin}{ 1cm}  % 3.17cm - 1 inch
\setlength{\evensidemargin}{\oddsidemargin}
\setlength{\textwidth}{13.50cm}
\vspace{-.8cm}
\begin{center}
\parbox{\textwidth}{
\CJKfamily{hei}摘~~~要\quad \CJKfamily{kai}~本文将利用Matlab生成钢琴曲,本文将先对于网上已有的钢琴音频进行分析,记录了
基频频率以及前四个谐频的幅值。然后通过相对应的方式反过来合成对应的曲调。最后通过这些曲调实现一个完整的曲子。\\
\CJKfamily{hei}关键词\quad\CJKfamily{kai}模拟乘法器,JFET管,平方非线性\\}
\end{center}

%\vspace{.1cm}
%\begin{center}
%\parbox{\textwidth}{
%{\large{\textbf{A Four-Quadrant Analog multiplier using JFET Transistors working in saturation region}}}\\
%\vspace{-0.5cm}
%\begin{center}
%\textbf{Wu Tianbo}\\[2pt]
%\small{\textit{(Dept. Automation, Tsinghua Univ.,China)}}\\[2pt]
%\end{center}
%{\small{\textbf{Abstract}\quad A novel circuit of analog multiplier is presented in this article.This is based on the square-law
%characteristics of the JFET transistors.It shows a quite well linearity for the inputs swing between -1V and +1V\\
%\textbf{Key Words}\quad Analog Multiplier, JFET, square-law characteristics}}
%}
%\end{center}

%\begin{minipage}[c]{10cm}
%\vspace{-35.5cm}
%文章编号~~~~1005$-$0388(2004)05$-$0505$-$04
%\end{minipage}

\setlength{\oddsidemargin}{-.5cm}  % 3.17cm - 1 inch
\setlength{\evensidemargin}{\oddsidemargin}
\setlength{\textwidth}{17.00cm}
\CJKfamily{song}
\begin{multicols}{2}
\section{引言}
\indent Matlab是一个强大的数学软件。在对于信号的处理上Matlab也有极为出色的表现。音乐作为一个十分有意义的声音信号也是
经常被用于在信号分析上。本文将会介绍利用Matlab分析已有的钢琴标准音阶的频谱特性,并通过这个特性反向还原出钢琴的音符,并利用这音符进行作曲。\\
\indent 我们在本文将首先介绍有关钢琴音阶的基本原理,介绍有关十二平均律的基本知识。之后本文将介绍利用Matlab处理声音信号的基本方式,以及在
本文中采用的用于处理钢琴音的方法。之后再第二部分我们会介绍如何利用从处理钢琴音得到的频域信息中还原出钢琴音并能最终组成一部曲子。\\
\section{原理}


\subsection{有关钢琴和音乐的基本知识}

\normalsize
\indent 十二平均律是现代音乐理论体系的一个基本理论基础。其基本原理来说就是每相隔十二个音阶的音的基准频率相差2倍,并且每一个音阶相差的频率在对数坐标上是相等的,即每一级相差$2^{\frac{1}{12}$倍的频率。这十二个音阶内的音在钢琴中被称为一个八度。
可以用字母表示为A,B,C,D,E,F,G,对应于简谱的1-7。但是如果要区分不同乐器的音色,还要考虑到谐音,即基频的倍数频的频率上幅值上的特征情况。因此如果要考虑较好的完整记录下钢琴音的信息可以
通过从网上开源音乐网站上找到的一个八度内的标准钢琴音,用Matlab进行分析记录特征。然后对于升调和降调则是通过直接基频增倍或是基频减半的方式实现。这样就可以获得基本上全部的音调。

\subsection{Matlab处理声音信号}
\indent Matlab是一个强大的数学分析软件,同时对于声音文件也有非常好的支持接口。直接利用Matlab内带的读取函数即可读取对应的信号。
接下来对于读取进来的音频信号尽心处理时,选择利用fft方法进行处理从而得到对应的频谱信息。虽然从理论上认为幅值最大的频率应该是信号的
基频信息,但实际上由于谐波混叠以及噪音的影响,一般来讲是二倍频处的幅值最大。另一方面,fft对应的是离散时间傅里叶变化的快速算法,如果要
使的横轴直接反应真实的频率信号和幅度,还需要做对应的处理。设原始信号采样频率为$F_s$,fft的采样点为N。则横轴x与竖轴y应该进行如下的变化:
\begin{equation}
			x^{'}=x*\frac{F_s}{N}-F_s/2
			y^{'}=y/(N/2)
\end{equation}
变化后的图像大致如下
\begin{figurehere}
\centering
\includegraphics[width=6cm](../audio/piano_a.jpg)
\end{figurehere}

\normalsize
\indent 在JFET管的饱和区,其漏极电流与栅极源极电压差之间有如下的关系
				\begin{equation}
							I_d=K(V_{GS}-V_{GS(off)})^2
				\end{equation}
其中K为一常数,$V_{GS(off)}$为JFET管的截止电压,一般小于零。
对于如图片\ref{theory}所示的电路,vx与vy为所输入的电压。
流过$R_1$的电流为$I_1$,流过$R_3$的电流为$I_3$,假设各JFET管均工作在
饱和区则有
\begin{equation}
				I_1 = K(vx-vy-U_T)^2+K(vy-vx-U_T)^2
\end{equation}

\begin{equation}
				I_3 = K(vx+vy-U_T)^2+K(-vy-vx-U_T)^2
\end{equation}

其中$U_T=V_6+U_{GS(off)}$是一个调节量,将其尽量控制在JFET正常工作电压中点以提高最大不失真输出电压。
这样输出电压Uo=V3-V1有
\begin{equation}
			U_o=-(I_3-I_1)R=-4KRvxvy
\end{equation}
这就构成了一个模拟乘法电路。
其中,如果我们需要获得vx-vy,vx+vy等一系列vx与vy运算得到的量,可以通过运算放大器构成的运算电路
得到。最终的总体电路图如图\ref{final}所示。

\begin{figure*}
\centering
\includegraphics[width=12cm]{../picture/final.png}
\caption{总体设计} \label{final}
\end{figure*}
				
\\
\indent 对于我所提出这个的电路,特点就是原理相对简单,所用的器件也相对普通。
同时可以通过调节运放实现输入电压范围变大但是输出电压范围不变。

\section{仿真测试结果}
\begin{figure*}
				\centering
				\includegraphics[width=12cm]{../picture/linear.png}
				\caption{模拟乘法器线性特性} \label{linear}
\end{figure*}

\begin{figure*}
				\centering
				\includegraphics[width=12cm]{../picture/distort.png}
				\caption{模拟乘法器失真特性} \label{distort}
\end{figure*}


\begin{figure*}
				\centering
				\includegraphics[width=12cm]{../picture/fuzai.png}
				\caption{模拟乘法器载波} \label{fuzai}
\end{figure*}

\indent我们一开始测试JFET管的负载特性,得出$U_T=-2.78V$时有较大的输出电压范围。 

\indent 按照如图\ref{final}所示的电路进行仿真,首先进行了线性条件测试,测试为输入电压
均在正负1V之间变化时,结果如图\ref{linear}。可以看出对于这一区间的输入,电路的输出特性
很好的符合模拟乘法器的特性。

如图\ref{fuzai}所示的是乘法器作为实现载波作用时的仿真输出结果。输入一个是1kHz的三角波,
另一个是20kHz的正弦波,峰峰值均为0.8V。结果如图中所示,可以看出该模拟乘法器的调幅特性较为好。

如图\ref{distort}显示的是在vy为3V时的失真线,可以看出在此时模拟乘法器就有较大的失真了,
这是因为此时-vx-vy有可能小于$U_T$,导致JFET有可能已经不工作在饱和区了,从而出现如图所示的失真情况。








\section{和其他模拟乘法器的比对}

\indent 该模拟乘法器主要受到ZhenHua Wang\cite{bult1986cmos}所提出的模拟乘法器的基础上予以的改进。相比于
该电路特点是不需要使用差模信号,JFET相比于MOS管漏极电流更小,这样更加节省能源,同时也更加安全。利用
运放提供对应的栅极输入信号虽然相对稍微复杂一些,但却提供方便了调节输入范围。\\

\indent 与书上提供的基于二极管的变跨导电路相比,由于JFET管利用的是平方非线性特性,而二极管一般要使用
对数非线性,一般来讲二极管所需的近似更多一些,而JFET管所构成的这个电路则相对不需要那么多的近似,所以会在
实现上更加的精确一些。\\

\indent然而本电路的缺点也是有一些的,首先是加入的运放部分使得本电路复杂性大幅提升,可能并不适合综合到实际
的应用电路中,作为研究的用处更多一点。第二,就是过于小的工作区间,由于JFET可工作区间有限,这使得输出电压无论在电路
的怎样的改进的情况下都是被局限在一个狭小的范围内的。\\

\section{结论}

\indent 本文提供了一个较为新颖的,利用JFET管的平方非线性特性的方式来实现的模拟乘法器。并且
对于该模拟乘法器实现了仿真,对于仿真的结果进行了一定程度上的分析。同时通过和其他已有的模拟乘法器进行了
对比,分析了该模拟乘法器的优点和不足之处。在这次研究过程中我体会了搜寻资料的不易,并且从模拟乘法器的发展历程
之中体会了总结前任工作,并在此基础上予以改进突破的重要性。





\end{multicols}

\clearpage
\end{CJK*}
\bibliographystyle{plain}%
\bibliography{bibfile}

\end{document}